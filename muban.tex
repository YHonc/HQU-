%用xlatex编译,其余部分可自行根据需求修改
%遵守开源协议,禁止商业用途
\documentclass[12pt,a4paper,oneside,UTF8]{ctexart}
%设置页边距
\usepackage[left=1.91cm,right=1.91cm,top=2.54cm,bottom=2.54cm]{geometry}
%需要用到的扩展包
\usepackage{xeCJK,amsmath,paralist,booktabs,multirow,graphicx,float,subfig,setspace,listings,xcolor}
\include{caption}
\usepackage{lastpage} % 确保加载 lastpage 宏包
\usepackage{hyperref}
\usepackage{fancyhdr}
\usepackage{algorithm}
\usepackage{algorithmic}

%去掉链接的红色外框
\hypersetup{
	colorlinks=true,
	linkcolor=black
}
%设置页眉页脚以及页码
\pagestyle{fancy}
\lhead{华侨大学}
\rhead{数学科学学院}
\cfoot{\thepage ~/~\pageref{LastPage}} % 使用 \pageref 替代 \zpageref
%\rfoot{\today} 

%报告中用到的图片存放在同级文件夹figures中,根据需求自行创建,若无需可删除该行
% \usepackage{graphicx} % 如果没有使用图形,可以删除这一行
\graphicspath{{figures/}}

%报告开始
\begin{document}
	%设置课程标题
	\begin{center}
		\heiti\LARGE{数值计算实验报告}
	\end{center}
	
	%设置实验人信息以及实验时间表格
	\begin{center}
		\begin{tabular}{lcr}
			{\songti 专~业:数学与应用数学} & {\songti 年~级:XXXX} &{\songti 班~级:XXXX} \cr
      {\songti 姓~名:XXX} & {\songti 学~号:XXXXXXXXXX} \cr
			{\songti{实~验~名~称:XXXXXX}} & {\songti{实~验~日~期:XXXXXXX}} \cr
		\end{tabular}
	\end{center}

	{\noindent}	\rule[-10pt]{17.18cm}{0.05em} %加一条分割线,长度为文本宽度(页面宽度减去两边的空白宽度)
	
	%实验题目
	\begin{center}
		\LARGE\textbf{实验名称:XXXXXXX}
	\end{center}
	
	%实验目的
	%\subsection*中的*代表[实验目的]这一标题不含有章节编号,\subsubsection*的*也是这一效果
	\subsection*{一、实验目的}
	\begin{enumerate}
		\item 示例1
		\item 示例2
		\item 示例3
	\end{enumerate}
	
	
	
	\subsection*{二、实验内容和步骤}
	填写实验题目
	

  \subsection*{三、设计思想}
  将实验所涉及的基础理论、算法原理详尽列出。
  \begin{enumerate}
  	\item 理论展示 : U分解是“矩阵因式分解”的一种,旨在将某个矩阵
    表示为两个或多个矩阵的乘积。正如其名,LU分解是将矩阵
    表示为
    ,其中矩阵
    代表Lower Triangular(下三角矩阵),矩阵
    代表Upper Triangular(上三角矩阵)。
  \end{enumerate}

  \begin{algorithm}[H]
    \caption{示例算法:求数组最大值}
    \label{alg:example}
    \begin{algorithmic}[1]
    \REQUIRE 一个包含 $n$ 个元素的数组 $A$
    \ENSURE 数组 $A$ 的最大值
    \STATE 初始化 $max \gets A[1]$
    \FOR{$i \gets 2$ 到 $n$}
        \IF{$A[i] > max$}
            \STATE $max \gets A[i]$
        \ENDIF
    \ENDFOR
    \RETURN $max$
    \end{algorithmic}
    \end{algorithm}
  
  
  \subsection*{四、实验程序}
列出实验的实施方案、步骤、数据准备、算法流程图以及可能用到的实验设备(硬件和软件)。\\
  \begin{itemize}
  \item 实验方案:XXXXXXXXXX
  \end{itemize}


  \begin{itemize}
      \item 实验步骤:
      \begin{itemize}
          \item 步骤1
          \item 步骤2
          \item 步骤3
      \end{itemize}
      \item 数据准备:
      \begin{itemize}
          \item 数据1
          \item 数据2
          \item 数据3
      \end{itemize}
      \item 算法流程图:见图~\ref{fig:test1}。
  \end{itemize}
\begin{figure}[H]
    \centering
    \includegraphics[width=0.8\textwidth]{figures/test1.jpg}
    \caption{算法流程图示例}
    \label{fig:test1}
  \end{figure}

  \begin{itemize}
    \item 程序代码:
    \begin{lstlisting}[language=matlab, caption=示例代码1, label={}]
      dashd
      dadw
      dadwxc
      sd
    \end{lstlisting}

    \begin{lstlisting}[language=matlab, caption=示例代码2, label={}]
      dashd
      dadw
      dadwxc
      sd
    \end{lstlisting}

  \end{itemize}


  \begin{itemize}
      \item 实验设备:
      \begin{itemize}
  	\item 硬件:XXXXXX
  	\item 软件:XXXXXX
      \end{itemize}
  \end{itemize}

	\subsection*{五、实验算例和结果}
  实验结果应包括试验的原始数据、中间结果及最终结果,
  复杂的结果可以用表格或图形形式实现,较为简单的结果可以与实验结果分析合并出现。\\
  \begin{itemize}
  	\item 算例1:XXXXXX
  	\begin{itemize}
  		\item 结果1:XXXXXX
  		\item 结果2:XXXXXX
  	\end{itemize}

  	\item 算例2:XXXXXX
  	\begin{itemize}
  		\item 结果1:XXXXXX
  		\item 结果2:XXXXXX
  	\end{itemize}
  \end{itemize}

  \subsection*{六、实验结果分析}
  实验结果分析应包括对实验结果的分析、对比、讨论等,  
  以及对实验结果的总结和归纳。\\  
  \begin{itemize}
  	\item 分析1:XXXXXX
  	\item 分析2:XXXXXX
  	\item 分析3:XXXXXX
  \end{itemize}

  \subsection*{七、实验出现的问题及体会}
  
  \begin{itemize}
    \item 问题1:XXXXXX
    \item 问题2:XXXXXX
  \end{itemize}

	{\noindent}	\rule[-10pt]{17.18cm}{0.05em} 

  \subsection*{\textbf{教师评语}}
  \vspace{2cm} % 第一行空白
  \vspace{2cm} % 第二行空白
  \vspace{2cm} % 第三行空白
  \vspace{2cm} % 第四行空白

  \begin{flushright}
    \textbf{指导教师:\underline{\hspace{3cm}}} \\ % 换行显示
    \textbf{\underline{\hspace{1.5cm}}年\underline{\hspace{1.5cm}}月\underline{\hspace{1.5cm}}日}
  \end{flushright}

\end{document}
% 在文档末尾添加以下内容以刷新页码
\clearpage
\label{LastPage} % 使用 \label 标记最后一页,用xlatex编译即可正确显示页码!!!第一行的页码会显示为??,第二次编译后会显示正确的页码(图片同理) 